\documentclass[12pt]{article}

\usepackage[brazil,american]{babel}
\usepackage[utf8]{inputenc}
\usepackage[a4paper, total={6.5in, 9.5in}]{geometry}

\usepackage{url}
\usepackage{graphicx}
\usepackage{authblk}
\usepackage{hyperref}
\usepackage{lipsum}
\usepackage{xcolor}
\usepackage{float}
\usepackage{listings}

\definecolor{codegreen}{rgb}{0,0.6,0}
\definecolor{codegray}{rgb}{0.5,0.5,0.5}
\definecolor{codered}{rgb}{0.8,0,0}
\definecolor{backcolour}{rgb}{0.95,0.95,0.92}

\usepackage{inconsolata}
\lstset{
    language=python,
    % backgroundcolor=\color{backcolour},   
    commentstyle=\color{codegreen},
    keywordstyle=\color{blue},
    numberstyle=\tiny\color{codegray},
    stringstyle=\color{codered},
    basicstyle=\ttfamily\small,
    % numberstyle=\footnotesize,
    % numbers=left,
    % backgroundcolor=\color{gray!10},
    % frame=single,
    tabsize=2,
    rulecolor=\color{black!30},
    title=\lstname,
    escapeinside={\%*}{*)},
    breaklines=true,
    breakatwhitespace=true,
    framextopmargin=2pt,
    framexbottommargin=2pt,
    inputencoding=utf8,
    extendedchars=true,
    showstringspaces=false,
    literate={á}{{\'a}}1 {ã}{{\~a}}1 {é}{{\'e}}1 {Ó}{{\'O}}1 {Ã}{{\~A}}1 {í}{{\'i}}1 {ó}{{\'o}}1,
}

% \pagecolor[rgb]{0.1,0.1,0.1}
% \color[rgb]{0.9,0.9,0.9}

\begin{titlepage}
    \title{
        \includegraphics[width=4cm]{img/logo.jpg} \\ 
        \large
        Dep. Ciência da Computação -- Universidade de Brasília (UnB)\\
        CIC0124 - Redes de Computadores, Turma A \\
        \vfill 
        \vfill
        \LARGE
        \textbf{Projeto Final\\
        Cliente DASH}
        \vfill
    }

    \author{
        Kesley Kenny Vasques Guimarães, 18/0021231\\
        Pedro Henrique de Brito Agnes, 18/0026305\\
        Victor Alves de Carvalho, 16/0147140
    }
    
    \affil{
        \vfill
        \vfill
        \vfill
        Professor \\
        Dr. Marcos Fagundes Caetano
    }

    \date{Brasília\\Dezembro de 2020}

\end{titlepage}

\begin{document}
\maketitle
\newpage

\selectlanguage{brazil}

\section{Introdução}
Para a implementação de um cliente DASH, existem diversos fatores a considerar. A largura de banda disponível aos usuários pode ser instável no geral, o que dificulta um serviço de \textit{streaming} de vídeo estável, ou seja, sem pausas ou de qualidade. Por isso, os serviços desse tipo atualmente disponibilizam diversas qualidades da imagem do vídeo diferentes, onde é atribuída a qualidade ideal, com o tempo necessário para baixar mais adequado à realidade da internet do usuário.

Outro conceito utilizado é o de \textit{buffer}, que é um local onde serão armazenados os segmentos de vídeo baixados para a reprodução ao usuário posteriormente. De maneira geral, os segmentos a serem reproduzidos ao usuário vêm do \textit{buffer}, que é útil para as flutuações de banda, onde no caso de o usuário experienciar uma queda na qualidade da internet, naturalmente vai demorar mais para obter um segmento de uma qualidade melhor, o que pode diminuir o tamanho do \textit{buffer}, mas são usados métodos para evitar ao máximo que o tamanho do mesmo chegue em 0, que resultaria em uma pausa indesejada no vídeo. Para o controle de todos os fatores variáveis incluídos em um cliente DASH, são usados Algoritmos de Adaptação de Taxa de Bits (ABR), que consistem em algoritmos que devem avaliar os recursos e estatísticas disponíveis de forma a buscar a melhor experiência ao usuário que está usando o serviço de vídeo.

\newpage

\section{Algoritmo ABR}
\lipsum[1]

\begin{lstlisting}
# codigos aqui
print("Hello world")
\end{lstlisting}

\section{Conclusões}
\lipsum[3]

\lipsum[5]

\lipsum[10]

\begin{thebibliography}{9}

\bibitem{livroRC}
\noindent James F. Kurose \& Keith W. Ross, 
\textit{Redes de Computadores e a Internet - Uma nova Abordagem, 7a /8a Edição, Pearson Education / Makron Books.}

\end{thebibliography}

\end{document}
